
\documentclass[11pt]{article}
\usepackage[margin=1in]{geometry}
\usepackage{enumitem}
\usepackage{hyperref}
\usepackage{xcolor}
\hypersetup{
    colorlinks=true,
    linkcolor=blue,
    urlcolor=blue
}

\title{CSCE 585: Machine Learning Systems}
\author{Pooyan Jamshidi}
\date{Fall 2025}

\begin{document}
\maketitle

\section*{Course Description}
Machine Learning Systems (MLSys) explores the intersection of machine learning, systems, and software engineering. 
Students will learn to critically read MLSys papers, design reproducible experiments, and implement scalable, efficient, 
and trustworthy ML systems. This version emphasizes LLM systems, agentic AI, and modern production challenges.

\section*{Learning Objectives}
By the end of the course, students will be able to:
\begin{itemize}[leftmargin=*]
    \item Critically evaluate MLSys research and impactful papers
    \item Design and analyze reproducible ML experiments
    \item Understand production challenges in ML (MLOps, LLMOps, monitoring)
    \item Explore modern frontiers: LLM serving, agentic AI, sustainability, privacy
    \item Communicate findings in technical reports and presentations
\end{itemize}

\section*{Tentative Weekly Schedule}
\begin{enumerate}[leftmargin=*]
    \item Lecture 1: Course Overview \& Motivation
    \item Lecture 2: ML in Production (MLOps foundations)
    \item Lecture 3: Designing Agentic AI Systems (patterns)
    \item Lecture 4: Case Study -- Designing LLM/Agentic Systems
    \item Lecture 5: LLMs in Production Systems (LLMOps, vLLM, Ray Serve)
    \item Lecture 6: Data, Evaluation \& Monitoring for LLM Systems
    \item Lecture 7: Efficiency \& Sustainability in ML Systems
    \item Lecture 8: Trustworthy \& Safe MLSys (robustness, governance)
    \item Lecture 9: Reproducibility \& Benchmarking in MLSys
    \item Lecture 10: What Makes an Impactful MLSys Paper
    \item Lectures 11--12: Capstone Prep, Guest Lectures, Student Presentations
\end{enumerate}

% \section*{Assignments \& Projects}
% \begin{itemize}[leftmargin=*]
%     \item Weekly paper critiques and discussions
%     \item Replication \& lab assignments (InferLine, vLLM, RAG, robustness, energy profiling)
%     \item Capstone project (motivation, design, final report \& presentation)
% \end{itemize}

% \section*{Grading Breakdown}
% \begin{tabular}{ll}
% Reading \& Critique Assignments & 15\% \\
% Replication \& Lab Assignments  & 30\% \\
% Capstone Project                & 40\% \\
% Participation \& Discussions    & 15\% \\
% \textbf{Total}                  & 100\% \\
% \end{tabular}

\end{document}
